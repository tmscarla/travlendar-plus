\section{Introduction}
\subsection{Purpose and Scope}
This Document represents the Implementation and Testing Document for the \textit{Travlendar+} project.\\
The purpose of this document is to provide a comprehensive overview of the implementation and testing activity for the development of the application.\\
\textit{Travlendar+} application aims at providing a calendar based system that allows users to schedule meetings and appointments, giving information concerning travel means and times to reach a specific event also providing options to book rides .

\subsection{Definitions, Acronyms, Abbreviations}

\subsubsection{Definitions}
\begin{itemize}
	\item \textbf{Framework}: is an abstraction in which software providing generic functionality can be selectively changed by additional user-written code.
	\item \textbf{Packet manager}: a software tool designed to optimize the download and storage of binary files, artifacts and packages used and produced in the software development process.
\end{itemize}

\subsubsection{Acronyms}

\begin{center}
	\begin{tabular}{| l | l |}
		\hline
		DBMS & Data Base Management System\\
		HTTP & Hyper Text Transfer Protocol\\
		HTTPS & Hyper Text Transfer Protocol Secure\\
		API & Application Program Interface \\
		REST & REpresentational State Transfer\\
		ORM & Object-Relational Mapping\\
		MVC & Model View Controller\\
		SDK & Standard Development Kit\\
		JSON & JavaScript Object Notation \\
		CRUD & Create, Read, Update and Delete\\
		LLVM & Low Level Virtual Machine\\
		CSRF & Cross-Site Request Forgery\\
		\hline
	\end{tabular}
\end{center}

\subsection{Reference Documents}
\begin{itemize}
	\item Design document
	\item RASD document
	\item Project assignment
\end{itemize}

\subsection{Overview}
The rest of the document is organized in this way:
\begin{itemize}
	
	\item \textbf{Implemented requirements:} explains which functional requirements outlined in the RASD are accomplished, and how they are performed.
	\item \textbf{Design choices:} provides reasons about the implementation decisions taken in order to develop the application.
	\item \textbf{Source code structure:} explains and motivates how the source code is structured both in the front end and in the back end.
	\item \textbf{Testing:} provides the main testing cases applied to the the application
	\item \textbf{REST API:} describes the API implemented for the application.
	
\end{itemize}
